\chapter{Literature Review}

The application of CNN-based models and data augmentation techniques in medical imaging has significantly advanced over the past decade, particularly in the domain of polyp
segmentation. This section synthesizes findings from contemporary research, focusing on
segmentation architectures, encoders, and augmentation strategies for improving segmentation performance [1, 2, 3, 4].
Colorectal cancer is one of the most common forms of cancer worldwide. It encompasses both colorectal and rectal cancers and poses a significant threat to health.
Colorectal cancer ranks as the third most common cancer, accounting for over 10 million
new cases annually [5]. In 2020, approximately 1.9 million new cases of colorectal cancer
were reported globally, representing 9.7% of all cancer cases [6].
One of the most effective methods for early detection of colorectal cancer is colonoscopy,
a procedure that allows doctors to visually inspect the inner walls of the intestines for
polyps and other abnormal changes. However, despite the high effectiveness of this method,
detecting polyps, which can be precursors to cancer, remains a challenging task [2]. Polyps
can be small, poorly visible, or hidden due to the anatomical features of the colon, increasing the likelihood of their being missed during the procedure [3].
In recent years, with the advancement of computer vision and machine learning
technologies, new approaches have emerged that enhance diagnostic accuracy and reduce
the workload of specialists. In particular, the use of deep learning methods for segmenting
images obtained through colonoscopy allows for automatic and precise detection of polyps,
significantly speeding up the diagnostic process and improving its accuracy. These methods
are actively being researched and implemented in clinical practice, leading to improved
treatment outcomes and a reduction in colorectal cancer mortality rates [4].
Polyps segmentation in the intestines using artificial intelligence (AI) and deep
neural networks is a promising direction in medical diagnostics. Training models using
large annotated datasets enables the development of highly accurate algorithms capable of
not only detecting polyps but also classifying them based on their potential for malignant
transformation. This opens new horizons for personalized treatment and cancer prevention
strategies [7].
8
The application of deep learning techniques, especially neural networks, has gained
significant attention in the medical field, particularly in image analysis. Among the various
tasks in medical imaging, image segmentation has proven to be one of the most crucial
challenges, as it helps in delineating structures of interest within the image, such as organs,
tumors, or pathological lesions [8].
Deep learning-based segmentation methods, specifically convolutional neural networks (CNNs), have become the state-of-the-art in various medical imaging applications.
CNNs are highly effective due to their ability to automatically learn spatial hierarchies of
features, making them ideal for tasks like organ segmentation, tumor detection, and polyp
detection in colonoscopy images [9].
One of the key advantages of using deep learning for segmentation tasks is the
reduction in manual labor required for annotating medical images. This has not only
accelerated the pace of medical research but also enhanced diagnostic accuracy. Moreover, neural networks can assist clinicians by providing reliable results that complement
traditional methods, potentially leading to earlier detection of diseases such as cancer [10].
Several methods have been proposed for segmentation tasks in medical imaging,
including fully convolutional networks (FCNs), U-Net, and Mask R-CNN. These architectures have demonstrated exceptional performance in both 2D and 3D medical image
segmentation [11]. U-Net, in particular, has been extensively used in tasks involving organ
segmentation, such as in MRI or CT scans, due to its efficient use of both global context
and fine details [12].
Recent advances also involve the integration of multi-modal data, where models
combine information from different imaging modalities, such as CT, MRI, and endoscopic
images, to improve segmentation accuracy [13]. Furthermore, the advent of transfer learning has allowed neural networks to achieve high performance even with limited annotated
data, making deep learning models increasingly accessible for medical practitioners [14].
The potential of neural networks in medical segmentation is vast. However, challenges remain, including the need for high-quality annotated data, dealing with the variability of imaging data across different patients and devices, and ensuring the interpretability
and trustworthiness of deep learning models in clinical settings [15].
Despite the significant advancements in deep learning for medical image segmentation, several challenges remain. One of the primary issues is the variability of medical
images across different patients, scanners, and imaging modalities. This variation introduces significant noise and inconsistency in the data, making it difficult for segmentation
models to generalize well across diverse populations and imaging conditions [16]. Moreover, medical images, particularly those used for tasks like polyp detection, often suffer
9
from poor image quality due to factors like motion artifacts, suboptimal lighting, and
differences in the expertise of the imaging personnel [17].
A significant hurdle for deep learning in medical imaging is the scarcity of highquality annotated data. Annotating medical images is a time-consuming and resourceintensive process that often requires the expertise of medical professionals. As a result, the
availability of large, high-quality annotated datasets is limited, and many deep learning
models are trained on relatively small datasets. This issue is particularly pronounced
for tasks such as polyp segmentation, where obtaining accurate annotations for every
image is both challenging and expensive [18]. Furthermore, the lack of diverse datasets
in terms of patient demographics (age, gender, ethnicity) and clinical conditions can limit
the performance of models when applied to real-world scenarios [19].
To address these issues, several techniques have been proposed. One such method is
data augmentation, which involves artificially expanding the dataset by applying transformations such as rotation, scaling, and flipping to the images. This helps mitigate overfitting
and improve model robustness [20]. Another strategy is semi-supervised learning, where
models are trained using a small amount of labeled data and a larger volume of unlabeled
data [21]. Additionally, transfer learning has proven effective, allowing models pre-trained
on large, publicly available datasets to be fine-tuned on smaller medical datasets, thereby
improving generalization performance [14].
Several deep learning-based models have already been implemented for medical
image segmentation tasks, with some achieving state-of-the-art results in colorectal polyp
detection. Among the most prominent architectures are U-Net and its variants, such as
U-Net++, which have been extensively used in medical imaging due to their ability to
capture both global context and fine-grained details of images [22]. Another successful
implementation is DeepLab, a model that incorporates dilated convolutions to capture
multi-scale information, which is particularly useful for segmenting complex structures like
polyps in colonoscopy images [23].
In addition to the traditional segmentation tasks, AI-driven systems have also been
employed in robotic-assisted surgery. One example is the use of AI for polyp detection
during robot-assisted colonoscopy, where AI models guide the robot in real-time to identify
suspicious areas that require biopsy. These systems help improve the accuracy and speed of
procedures, reduce the cognitive load on surgeons, and increase the likelihood of detecting
early-stage cancer [24].
Another exciting application of deep learning in medicine is in robot-assisted surgeries, such as those performed by the Da Vinci Surgical System. The Da Vinci robot, which
is used for minimally invasive surgeries, can be enhanced with AI-driven image segmen10
tation techniques to automatically identify and highlight key structures like blood vessels,
tumors, or polyps. This real-time feedback aids surgeons in making precise decisions, ultimately improving patient outcomes and reducing surgical risks [25]. Additionally, the
development of robots like DaLi (Deep Learning Robot for Invasive Surgery) marks a significant step forward in integrating AI with robotic surgery. These robots are equipped with
advanced segmentation algorithms to identify critical anatomical structures in real-time,
assisting in complex procedures such as cancer removal, organ transplantations, and polyp
excisions during colonoscopies. The ability to precisely segment and classify structures in
medical images significantly enhances the robot’s performance, ensuring that surgeons can
make more informed decisions with less risk of human error [26].
In conclusion, while the application of deep learning techniques to medical image
segmentation has revolutionized the detection and treatment of diseases such as colorectal
cancer, several challenges remain. Issues related to data scarcity, variability in imaging
conditions, and the need for accurate annotations continue to hinder the widespread implementation of deep learning models in clinical settings. However, promising advancements
in data augmentation, semi-supervised learning, and transfer learning offer potential solutions to these problems [27].
Future work in the field should focus on improving the generalization ability of
models across diverse populations and clinical conditions. Additionally, the integration of
AI with other diagnostic tools, such as robotic systems, holds great promise for improving
the accuracy, speed, and efficiency of medical procedures. As more annotated datasets
become available and as model architectures continue to evolve, deep learning-based segmentation systems will likely become a standard component of medical imaging workflows,
contributing to earlier diagnosis and better treatment outcomes.
